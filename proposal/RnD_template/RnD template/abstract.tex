\begin{abstract}
Energy efficient vehicle controller have the purpose to reduce the energy consumption of vehicles. Using such optimal controllers could help to save money and reduce CO2 emissions. But optimal control strategies are hard to come up with, even for known tracks and when computed offline. \\
Previous approaches to energy efficient vehicle control include optimal control theory and graph search algorithms. Unfortunately solutions using optimal control theory can only be found under certain conditions and graph search algorithms need very much space for coming up with good approximations to the optimal solution.\\
A recent approach proposes the use of artificial neural networks and evolutionary strategies to come up with an optimal control strategy. One advantage of this approach is that the resulting strategies do not need much space. 
So far this approach has been applied only in simulation. 
But literature showed that transferring evolved solutions from simulation into reality usually yields suboptimal results (called "Reality Gap"). \\
There have been approaches to address the Reality Gap problem in other domains, including the addition of noise, the use of more accurate models for simulation and the so-called transferability approach.\\
The target of this R\& D is the real-world validation of an energy efficient vehicle controller designed by using evolutionary strategies and the application of methods to overcome the expected Reality Gap. 
\end{abstract}
